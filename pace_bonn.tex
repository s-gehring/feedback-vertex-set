\documentclass[11pt,a4paper]{scrartcl}
\usepackage[english]{babel}

\begin{document}
\title{Documentation}
\subtitle{PACE 2016 - Track B: Feedback Vertex Set}
\author{Fabian Brand, Simon Gehring, Florian Nelles,\\ Kevin Wilkinghoff, Xianghui Zhong\\
	supervised by: Stefan Fafianie, Stefan Kratsch}
\maketitle

\section{Degree 3 Case}

We detect and solve the degree 3 special case in polynomial time as described in \cite{DBLP:journals/corr/abs-1004-1672}. Whenever we detect in the iterated compression that all vertices in $V_1$ have degree at most 3, we first delete all vertices with degree 1 or less that were created by the branching process and then consider every connected component seperately. We reduce this special case to the cographic matroid parity problem of the auxility graph described in Section 3 of the paper. To solve this problem, we reduce it to the colinear matroid parity problem by generating the incidence matrix and transfer it to the linear parity problem. Here the entries of the matrix are taken in finite fields to allow for exact calculations. In the end we solve the linear parity problem by \cite{Cheung:2014:AAL:2620785.2601066}. The idea is based on the theorem of Lovasz. The galois fields have size $2^{16}$, so by the .

\addcontentsline{toc}{section}{literature}
\nocite{bafna1999}
\nocite{chen2008}
\bibliography{literature}
\bibliographystyle{abbrv}

\end{document}
